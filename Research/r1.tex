\documentclass{article}
%\usepackage[utopia]{mathdesign}
\usepackage{cmbright}
\usepackage[UKenglish]{isodate}
\cleanlookdateon
\usepackage[UKenglish]{babel}
\usepackage[T1]{fontenc}
\usepackage{amsmath}
\usepackage{lettrine}
\usepackage{lipsum}
\usepackage{fancyhdr}
\pagestyle{fancy}
\fancyhead[R]{\today}
\fancyhead[L]{Game of Clones}
\fancyhead[C]{Research Paper}
\usepackage{sectsty}
\sectionfont{\noindent\normalfont}
\usepackage{microtype}

\begin{document}
\title{\Huge{Game of Clones}}
\author{Clement Lee\\
West High\and
Oliver Richardson\\
West High}
\maketitle

\begin{abstract}
Game theory and economics provide simple methods for investigating the principles behind decision making, which are themselves the basis for societies. However, cooperation, as a vital foundation of society, is often a complex variable that is hard to model correctly, as the expected value of the problem is not well-defined. The goal of this computer science project is to provide a generalized simulation that allows testing of environmental factors on the sustainability and viability of cooperation. A model for each agent within the games has been developed to fuse particle swarm optimizers and genetic algorithms, emulating many factors of biology. These agents all have to act within a world defined by two-player and cooperative games, where decisions are made to either defect or cooperate. These games, and the agents themselves, are controlled by a set of environmental variables, all of which are randomized within a range to guarantee a varied but controllable world. Furthermore, this simulation is in itself automated, applying a meta-particle swarm optimizer on the environment variables, with the intent of simplifying the huge amounts of data generated into a simpler, more comprehensible form. The overall objective of the project is to determine practical knowledge that can be applied to modern-day social situations, where cooperation is immensely important for conflict resolution. Future work on this project will be in giving complexity to the agents through the use of alternative decision-making methods, as well as introducing more types of environment and agent variables to more accurately simulate reality.
\end{abstract}

\section*{Background\dotfill}
\lettrine{E}{conomics} is the study of choices made by individuals under a variety of situations. In order to enumerate and analyze these choices to maximize benefit, we use game theoretical concepts to define 

\section*{Question\dotfill}
\section*{Hypotheses\dotfill}
\section*{Algorithms\dotfill}
\section*{Coding Practices\dotfill}
\section*{Conclusion\dotfill}
\section*{Further Research\dotfill}

\begin{thebibliography}{1}

\end{thebibliography}
\end{document}